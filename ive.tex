\documentclass[aspectratio=1610,xcolor=dvipsnames,t]{beamer} 

\usepackage{listings} 
\usepackage{color} 
\usepackage{xcolor}  
\usepackage{microtype} 
\usepackage{helvet} 
\usepackage{inconsolata} 
\usepackage[framemethod=TikZ]{mdframed} 
\usepackage{graphicx} 
\usepackage{alltt}
\usepackage{sverb} 
\usepackage{verbatim} 
\usepackage{pifont} 
\usepackage{helvet} 
\usepackage{algorithm}
\usepackage{algpseudocode}

\usetheme{Madrid} 
\useinnertheme{rectangles} 

\setbeamertemplate{navigation symbols}{}
\setbeamertemplate{blocks}[default] 

%\definecolor{mypurple}{rgb}{.49,0,98}
%\setbeamercolor*{palette primary}{use=structure,fg=white,bg=green}
%\usecolortheme[rgb={0.9,0.2,0.2}]{structure}
%\usecolortheme[rgb={0.6,0.1,0.1}]{structure}

%\usecolortheme[rgb={0.2, 0.2, 0.8}]{structure} 
%\usecolortheme[rgb={0.0, 0.0, 0.8}]{structure} 
\usecolortheme[rgb={0.2, 0.2, 0.3}]{structure} 

\usepackage{color}
\definecolor{orange}{cmyk}{0,0.4,0.8,0.2}
\definecolor{darkorange}{rgb}{.71,0.21,0.01}
\definecolor{darkgreen}{rgb}{.12,.54,.11}
\definecolor{myteal}{rgb}{.26, .44, .56}
\definecolor{gray}{gray}{0.45}
\definecolor{lightgray}{gray}{.95}
\definecolor{mediumgray}{gray}{.8}
\definecolor{inputbackground}{rgb}{.95, .95, .85}
\definecolor{outputbackground}{rgb}{.95, .95, .95}
\definecolor{traceback}{rgb}{1, .95, .95}
\definecolor{inputbg}{rgb}{0.98, 0.98, 0.98}

\usepackage{listings} 
\lstset{language=bash,
        %basicstyle=\footnotesize\ttfamily, 
        basicstyle=\small\ttfamily,
        columns=fullflexible, 
        %title=\lstname, 
        %numbers=left, stringstyle=\texttt, 
        %numberstyle={\tiny\texttt}, 
        keywordstyle=\color{blue}, 
        commentstyle=\color{darkgreen}, 
        stringstyle=\color{purple} } 


\mdfsetup{skipabove=\topskip, skipbelow=\topskip} 

\definecolor{codebg}{rgb}{0.99,0.99,0.99}

\global\mdfdefinestyle{code}{%
    frametitlerule=true,%
    frametitlefont=\small\bfseries\ttfamily,%
    frametitlebackgroundcolor=lightgray,%
    backgroundcolor=codebg,%
    linecolor=gray, linewidth=0.5pt,%
    leftmargin=0.5cm, rightmargin=0.5cm,%
    roundcorner=2pt,%
    innerleftmargin=5pt
}

\global\mdfdefinestyle{code2}{%
    topline=false,%
    bottomline=false,%
    leftline=true,%
    rightline=false,%
    backgroundcolor=codebg,%
    linecolor=gray, linewidth=0.5pt,%
    leftmargin=0.0cm, rightmargin=0.0cm,%
    innerleftmargin=1pt
}

\newcommand{\showcode}[1]{\begin{mdframed}[style=code] %
                            \lstinputlisting{#1}% 
                          \end{mdframed}% 
}


\title[Intelligent Virtual Environments]{Intelligent Virtual Environments for Agents} 
\subtitle{2002 Australian Cognitive Science Conference} 
\author[Michael Papasimeon]{Michael Papasimeon} 
\date[OZCOGSCI 2002]{25 September 2002\\ Fremantle, Western Australia} 

\begin{document}

\begin{frame}
    \maketitle
\end{frame} 

\begin{frame}{Overall Aim and Hypothesis} 
    \begin{block}{Aim}
        To allow agents to participate in a richer, more complex and more
        intelligent way in their environment in the framework of an explainable
        and plausible cognitive model. 
    \end{block}
    \begin{block}{Hypothesis}
        \begin{itemize}
            \item Current agents are limited by their environmental
                  interaction.
            \item We can attempt to change this by improving the way
                  in which agents interact with their environment.
        \end{itemize}
    \end{block} 
    \begin{exampleblock}{Herb Simon (1969)} 
        \begin{quote}
            "Complexity of an ant's behaviour walking along a beach
            has more to do with the complexity of the environment
            rather than an inherent internal complexity of the 
            ant itself."
        \end{quote}
    \end{exampleblock} 
\end{frame} 

\begin{frame}{Definitions: Agent} 
    \begin{block}{Russell and Norvig \emph{Artificial Intellligence -- pg 31, 1995}}
        \begin{quote}
            "An agent is anything that can be viewed as perceiving its 
            \textbf{environment} through sensors and acting upon that
            \textbf{environment} through effectors."
        \end{quote}
    \end{block} 

    \begin{block}{d'Inverno and Luck \emph{Understanding Agent Systems -- pg 2, 2001}}
        \begin{quote}
            "... agents have been proposed as \textbf{situated} and 
            \textbf{embedded} problem solvers that are capable of functioning
            effectively and efficiently in a complex \textbf{environment}."
         \end{quote} 
    \end{block} 

    \begin{block}{Wooldrdige \emph{Multiagent Systems - pg 29, 1999 - editor G. Weiss}}
        \begin{quote}
            "An agent is a computer system that is \textbf{situated} in some
            \textbf{environment} and that is capable of autonomous 
            \textbf{action} in this environment in order to meet its
            design objectives."
        \end{quote}
    \end{block} 

\end{frame} 

\begin{frame}{Definition: Environment}
\end{frame} 

\begin{frame}{Requirement for Virtual Environments} 
\end{frame} 

\begin{frame}{Definition: Intelligent}
\end{frame} 

\begin{frame}{Designing Intelligent Information Systems} 
\end{frame}

\begin{frame}{Real Environments: Augmentation} 
\end{frame} 

\begin{frame}{Classical AI vs Situated Cognition}
\end{frame} 

\begin{frame}{Virtual Environments}
\end{frame}

\begin{frame}{Motor Racing Simulation: The Scenario}
\end{frame} 

\begin{frame}{Environmental Representation Options} 
\end{frame} 

\begin{frame}{Intelligent Agent: Low Level Perception}
\end{frame} 

\begin{frame}{An Environment Labelled for an Agent}
\end{frame} 

\begin{frame}{Driver Agent: Rounding a Corner}
\end{frame} 

\begin{frame}{Labels, Names, Categories and Plans}
\end{frame} 

\begin{frame}{Environmental Labelling by Category} 
\end{frame}

\begin{frame}{Relationships in the Environment}
\end{frame}

\begin{frame}{Affordances in Crazy Taxi} 
\end{frame} 

\begin{frame}{An Environment Labelled for an Agent} 
\end{frame} 

\begin{frame}{Summary}
\end{frame} 


\end{document}
